\documentclass[]{article}

\usepackage{verbatim}

%opening
\title{Automated Benchmarking \& Comparison of Amazon EC2 Instances}
\author{Pierre Ballif, Matyáš Ješina, Jakub Profota}

\begin{document}

\begin{comment}
In your report, answer the following questions:
•1) Flask Application Deployment Procedure
•2) Cluster setup using Application Load Balancer
•3) Results of your benchmark
•4) Instructions to run your code

Your assignment will be graded on content as following:
•2 pts: Description of the environment setup and all necessary steps;
•2 pts: Validation of the environment setup;
•6 pts: Results and analysis of your benchmarks;
•3 pts: Automated benchmarking;
•6 pts: Demo;
•1 pts: General presentation and the quality of the report. It is important to
respect the format of the submission. Please use LaTeX!
\end{comment}

\maketitle

\begin{abstract}

\end{abstract}

\section{Deployment Procedure}

There are five resources that need to be deployed:
\begin{itemize}
	\item Create a security group allowing the resources to be accessed from port 22 (SSH) and port 80 (HTTP).
	\item Start eight EC2 instances, four of each type. Deploy a Flask app on each on them.
	\item Create two target groups, and register four instances with each target group.
	\item Create a load balancer.
	\item Set two rules for the load balancer: redirect requests for /cluster1 to the first and requests for /cluster2 to the second target group.
\end{itemize}

Due to the dependencies between these resources, they need to be created in exactly that order and shut down in the reverse order.

In order to deploy the Flask app on all the instances, we used the userScript parameter of the Amazon API. This allows to give the instance a script that will be run at startup. Our script installs all necessary packages, create an app.py file for Flask, and deploys a Flask server on port 80.

\section{Benchmarking}

\section{Results}

\section{Installation steps}

\begin{enumerate}
	\item Clone our Git repository and navigate into it: \\
	\verb|git clone git@github.com:jesinmat/LOG8415_cloud_TP1.git && cd LOG8415_cloud_TP1|
	\item Configure your credentials: get your credentials by logging in to Vocareum, then paste them into \verb|~/.aws/credentials|
	\item Deploy the Amazon resources: \verb|python3 load_balancer.py|
\end{enumerate}

\end{document}
